\documentclass{article}

\usepackage[czech]{babel}
\usepackage[utf8]{inputenc}
\usepackage[T1]{fontenc}
\usepackage{amsmath}
\usepackage{amsfonts}
\usepackage{physics}
\usepackage{enumerate}
\everymath{\displaystyle}
\begin{document}

{\centering {\bfseries
  Aplikovaná matematika, kombinovaná forma, pracovní list 1.}

  Odevzdat nejpozději 20.3., za správné řešení 3 body ke zkoušce.
  
  }


  
  \bigskip
Teplota ve dvourozměrné desce pro $0\leq x\leq 10$ a $0\leq y\leq 10$ je popsána rovnicí
  $$T(x,y)=(x+2y)^2+x^3.$$
  Rozměry jsou v centimetrech, teplota ve stupních Celsia. (Formálně to nevychází, ale ke každému členu můžeme dodat konstantu, která rozměr opraví tak, aby výsledek opravdu vycházel ve stupních Celsia. Pro jednoduchost tuto komplikaci vynecháme.)

  \bigskip
  \hrule
  
\begin{enumerate}[1)]
\item Vypočtěte parciální derivace $\pdv{T}{x}$ a $\pdv{T}{y}$.
\item Vypočtěte parciální derivaci $\pdv{T}{x} (2,1)$ v bodě $(2,1)$ (včetně jednotky) a podejte její slovní interpretaci.
\item Vypočtěte gradient $\nabla T$ v obecném bodě $(x,y)$.
\item Je dán součinitel tepelné vodivosti $\lambda=
  \begin{pmatrix}
    5 & 1\\1&4
  \end{pmatrix}.
$ Vypočtěte tok tepla $$-\lambda \cdot \nabla T.$$
\item Určete, zda na levém okraji desky teče teplo dovnitř desky nebo z desky ven. Poznáte to podle znaménka $x$-ové souřadnice toku tepla pro $x=0$.
\item Vypočtěte divergenci toku tepla, tj. $$\nabla\cdot(-\lambda \cdot \nabla T).$$
\item V desce nejsou zdroje tepla. Proto je situace modelována rovnicí
$$\rho c\pdv{T}{t}=\nabla\cdot(\lambda \cdot \nabla T).$$
  Ochlazuje se deska uprostřed, nebo otepluje?
\end{enumerate}

\hrule
\bigskip

\textbf{Poznámka:} V předposledním kroku je výsledek $$\nabla\cdot(-\lambda \cdot \nabla T)=-30x-50.$$ Pro odpověď na poslední otázku použijte rovnici vedení tepla. Znaménko z pravé strany určíte díky výsledku předchozího příkladu. Pozor na to, že v rovnici vedení tepla není minus na pravé straně.


\end{document}
