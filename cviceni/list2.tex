\documentclass{article}

\usepackage[czech]{babel}
\usepackage[utf8]{inputenc}
\usepackage[T1]{fontenc}
\usepackage{amsmath}
\usepackage{amsfonts}
\usepackage{physics}
\usepackage{enumerate}
\usepackage[margin=1in]{geometry}
\everymath{\displaystyle}
\begin{document}

{\centering {\bfseries
  Aplikovaná matematika, kombinovaná i prezenční forma.}

  }


  
  \bigskip
  \hrule
  
  \bigskip Uvažujte vektorové pole
  $$\vec F=(x+3y)\vec \imath + ax\vec\jmath.$$ Cílem je stanovit, kdy
  integrál tohoto vektorového pole nezávisí na integrační cestě,
  vypočítat integrál po třech různých křivkách, které mají stejný
  počáteční a koncový bod, najít kmenovou funkci a vypočítat křivkový
  integrál pomocí této kmenové funkce.  \bigskip \hrule
  
\begin{enumerate}[1)]
\item Ukažte, že jediná hodnota reálného parametru $a$ pro kterou integrál zadaného vektorového pole nezávisí na integrační cestě je $a=3$. Dále pracujte s touto hodnotou parametru $a$.
\item Vypočtěte hodnotu křivkového integrálu $$\int_{C_1} \vec F\mathrm d\vec r$$ po úsečce $\vec r(t)=t\vec \imath+t\vec \jmath$, $t\in[0,1]$.
\item Vypočtěte hodnotu křivkového integrálu $$\int_{C_2} \vec F\mathrm d\vec r$$ po části paraboly $\vec r(t)=t\vec \imath+t^2\vec \jmath$, $t\in[0,1]$. Ověřte, že výsledek je stejný, jako v předchozím případě.
\item Vypočtěte hodnotu křivkového integrálu $$\int_{C_3} \vec F\mathrm d\vec r$$ po křivce $\vec r(t)=t^2\vec \imath+t^3\vec \jmath$, $t\in[0,1]$. Ověřte, že výsledek je stejný, jako v předchozích případech.
\item Najděte funkci $\varphi(x,y)$ takovou, že $\nabla \varphi=\vec F.$
\item Použijte předchozí výsledek k výpočtu křivkového integrálu druhého druhu vektorového pole $\vec F$ po libovolné křivce z bodu $[0,0]$ do bodu $[1,1]$.
\end{enumerate}

\hrule
\bigskip

\begin{enumerate}[1)]
\item
  Pro nezávislost na integrační cestě musí platit $$\frac{\partial}{\partial x}(ax)=\frac{\partial}{\partial y}(x+3y)$$
  tj. $$a=3.$$ Vidíme, že nezávislost na integrační cestě bude zaručena pro jedinou hodnotu, $a=3$.
\item Derivací křivky $\vec r=(t,t)$ podle $t$ dostáváme
$$\frac{\mathrm d\vec r}{\mathrm dt}=(1,1).$$
Rovnice vektorového pole $\vec F(x,y)=(x+3y,3x)$ podél křivky má tvar
$$\vec F(\vec r(t))=(t+3t,3t)=(4t,3t).$$
Skalárním součinem dostáváme
$$\vec F \frac{\mathrm d\vec r}{\mathrm dt}=
(4t,3t)\cdot (1,1) = 4t+3t =7t.
$$
Odsud formálně $$\vec F\mathrm d\vec r=7t\mathrm dt$$
a integrál má tvar
$$\int_C\vec F\mathrm d\vec r=\int_0^{1}7t\,\mathrm dt=\left [\frac 72 t^2\right]_0^1=\frac 7 2.$$

\item Derivací křivky $\vec r=(t,t^2)$ podle $t$ dostáváme
$$\frac{\mathrm d\vec r}{\mathrm dt}=(1,2t).$$
Rovnice vektorového pole $\vec F(x,y)=(x+3y,3x)$ podél křivky má tvar
$$\vec F(\vec r(t))=(t+3t^2,3t).$$
Skalárním součinem dostáváme
$$\vec F \frac{\mathrm d\vec r}{\mathrm dt}=
(t+3t^2,3t)\cdot (1,2t) = t+3t^2+6t^2 =t+9t^2.
$$
Odsud formálně $$\vec F\mathrm d\vec r=(t+9t^2)\mathrm dt$$
a integrál má tvar
$$\int_C\vec F\mathrm d\vec r=\int_0^{1}t+9t^2 \,\mathrm dt=\left [\frac 12 t^2+3t^3\right]_0^1=\frac 12 +3=\frac 7 2.$$


\item Derivací křivky $\vec r=(t^2,t^3)$ podle $t$ dostáváme
$$\frac{\mathrm d\vec r}{\mathrm dt}=(2t,3t^2).$$
Rovnice vektorového pole $\vec F(x,y)=(x+3y,3x)$ podél křivky má tvar
$$\vec F(\vec r(t))=(t^2+3t^3,3t^2).$$
Skalárním součinem dostáváme
$$\vec F \frac{\mathrm d\vec r}{\mathrm dt}=
(t^2+3t^3,3t^2)\cdot (2t,3t^2) = 2t^3+6t^4+9t^4 =2t^3+15t^4.
$$
Odsud formálně $$\vec F\mathrm d\vec r=(2t^3+15t^4)\mathrm dt$$
a integrál má tvar
$$\int_C\vec F\mathrm d\vec r=\int_0^{1}2t^3+15t^4 \,\mathrm dt=\left [\frac 12 t^4+3t^5\right]_0^1=\frac 12 +3=\frac 7 2.$$

\item Platí
  $$\varphi (x,y)=\int x+3y\,\mathrm dx=\frac 12 x^2+3xy+C_1(y)$$
  a
  $$\varphi (x,y)=\int 3x\,\mathrm dy=3xy+C_2(x).$$
  Porovnáním
  $$\varphi (x,y)=\frac 12 x^2+3xy+C,$$
  kde $C$ je libovolná reálná konstanta.
\item Přímým výpočtem dostáváme
  $$\varphi(1,1)-\varphi(0,0)=\frac 12 + 3 - 0 =\frac 72$$
  a tato hodnota se shoduje se všemi vypočtenými integrály.

\end{enumerate}
\end{document}
